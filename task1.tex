\documentclass{article}
\usepackage{geometry}
\usepackage{booktabs}
\usepackage{multirow}
\usepackage{amsmath}
\usepackage{amssymb}
\usepackage{hyperref}

\title{Benchmarking CRYSTALS-Dilithium vs. Traditional Digital Signature Schemes: A Comparative Analysis in Terms of Space, Time, and Additional Metrics}
\author{}
\date{}

\begin{document}

\maketitle

\begin{abstract}
This report provides a comprehensive benchmark comparing the CRYSTALS-Dilithium digital signature algorithm—a leading post-quantum signature candidate—with traditional digital signature schemes such as RSA and ECDSA. We analyze the schemes in terms of space costs (public key and signature sizes), time costs (software cycle counts and hardware latencies), and additional metrics (hardware efficiency, side-channel resistance, and security margins). More than 20 references from the literature support this analysis. Our results show that although classical schemes may offer smaller sizes and sometimes lower latencies, their vulnerability to quantum attacks renders them unsuitable for long-term security. In contrast, Dilithium, with competitive performance and quantum resistance, represents a promising candidate for future secure communications.
\end{abstract}

\section{Introduction}
The advent of quantum computing poses a threat to widely deployed public-key cryptosystems such as RSA and ECDSA, whose security relies on hard problems (e.g., prime factorization and discrete logarithms) that are efficiently solved by Shor’s algorithm on a quantum computer \cite{ref5}. In response, post-quantum cryptographic algorithms have been developed; among these, CRYSTALS-Dilithium is a lattice-based digital signature scheme built upon module-LWE and module-SIS problems \cite{ref0, ref4}. Unlike traditional schemes, Dilithium is designed for constant-time implementations, which enhances its resistance against side-channel attacks.

This report benchmarks Dilithium against classical schemes by considering:
\begin{itemize}
    \item Space costs: public key and signature sizes.
    \item Time costs: measured as CPU cycle counts and hardware latencies.
    \item Additional metrics: hardware efficiency (area, power), side-channel resistance, and security margins.
\end{itemize}

\section{Benchmark Metrics and Methodology}

\subsection{Space Cost Metrics}
\begin{itemize}
    \item \textbf{Dilithium}: For example, its recommended (security level II) parameter set yields a public key of roughly 1.5 KB and a signature of approximately 2.7 KB \cite{ref0, ref4}.
    \item \textbf{RSA/ECDSA}: Traditional RSA (e.g., RSA-3072) typically uses several kilobytes for full key material and produces signatures of about 384 bytes, while ECDSA offers public keys around 256 bytes and signatures near 512 bytes \cite{ref5}.
\end{itemize}

\subsection{Time Cost Metrics}
\subsubsection{Software Performance (Cycle Counts/Latency)}
\begin{itemize}
    \item \textbf{Dilithium}: Optimized software implementations report signing latencies on the order of 500K CPU cycles and verification latencies of approximately 175K cycles on platforms such as the Intel Haswell processor \cite{ref0, ref2}.
    \item \textbf{Traditional Algorithms}: RSA signing is computationally heavy due to modular exponentiation, and while ECDSA signing is generally faster, both rely on complex number-theoretic operations \cite{ref5}.
\end{itemize}

\subsubsection{Hardware Implementations}
\begin{itemize}
    \item Recent FPGA and ARM implementations of Dilithium have demonstrated very competitive performance with low latency and modest area footprints \cite{ref2, ref6, ref8}.
    \item In many cases, hardware implementations of classical algorithms are well optimized; however, the quantum vulnerability of RSA/ECDSA remains a critical drawback.
\end{itemize}

\subsection{Additional Metrics}
\subsubsection{Security Margin}
\begin{itemize}
    \item Dilithium is based on lattice problems (module-LWE/module-SIS) that have been extensively studied and are believed to remain hard even in the presence of quantum computers. This is in contrast to the number-theoretic problems underlying RSA and ECDSA, which are known to be breakable by quantum algorithms such as Shor’s \cite{ref5, ref11}.
\end{itemize}

\subsubsection{Side-Channel Resistance}
\begin{itemize}
    \item The design of Dilithium explicitly avoids complex operations (like high-precision Gaussian sampling in some variants) in favor of uniform sampling and simple rejection steps, which eases constant-time implementation and improves resistance to side-channel leakage \cite{ref0}.
\end{itemize}

\subsubsection{Hardware Efficiency (Area, Energy, Memory Footprint)}
\begin{itemize}
    \item FPGA and ASIC implementations of Dilithium have shown low area and power consumption when compared with RSA implementations that require multiple-precision arithmetic units \cite{ref2, ref6}.
    \item Optimized vectorization (for instance, using AVX-512) further improves the time/space efficiency of Dilithium implementations \cite{ref9}.
\end{itemize}

\subsubsection{Comparative Analysis with Other Post-Quantum Schemes}
\begin{itemize}
    \item Studies comparing Dilithium with alternative post-quantum signature schemes (such as Falcon, SPHINCS+, and Rainbow) highlight trade-offs in key/signature sizes and computation times, placing Dilithium in a “sweet spot” for many practical applications \cite{ref1, ref3}.
\end{itemize}

\section{Benchmark Results}

\subsection{Space Costs}
Table~\ref{tab:space} shows the public key and signature sizes for each scheme.

\begin{table}[h!]
\centering
\caption{Space cost comparison of signature schemes.}
\label{tab:space}
\begin{tabular}{lcc}
\toprule
\textbf{Scheme} & \textbf{Public Key Size} & \textbf{Signature Size} \\
\midrule
Dilithium (Level II) & ~1.5 KB & ~2.7 KB \\
RSA (3072-bit) & Several KB & ~384 bytes \\
ECDSA (384-bit) & ~256 bytes & ~512 bytes \\
\bottomrule
\end{tabular}
\end{table}

\subsection{Time Costs}
Table~\ref{tab:time} compares the computational latencies.

\begin{table}[h!]
\centering
\caption{Time cost comparison (software cycle counts) of signature operations.}
\label{tab:time}
\begin{tabular}{lccc}
\toprule
\textbf{Operation} & \textbf{Dilithium} & \textbf{RSA} & \textbf{ECDSA} \\
\midrule
Key Generation & Moderate & Heavy & Light \\
Signing & ~500K cycles & Heavy & Moderate \\
Verification & ~175K cycles & Moderate & Fast \\
\bottomrule
\end{tabular}
\end{table}

\subsection{Additional Performance Metrics}
\begin{itemize}
    \item \textbf{Hardware Efficiency}: FPGA implementations of Dilithium show low area and energy consumption relative to RSA, which requires large multiprecision arithmetic units \cite{ref2, ref6}.
    \item \textbf{Side-Channel Resistance}: Dilithium’s design (via uniform sampling and simple rejection sampling) eases the design of constant-time implementations, reducing the risk of side-channel leakage \cite{ref0}.
    \item \textbf{Security Margin}: With its foundation in lattice problems, Dilithium maintains a high security margin in a post-quantum world, while RSA and ECDSA would be rendered insecure by quantum algorithms (e.g., Shor's) \cite{ref5, ref11}.
\end{itemize}

\section{Discussion and Trade-Off Analysis}
\begin{itemize}
    \item \textbf{Quantum Resistance vs. Efficiency}: Although RSA and ECDSA are well understood and optimized for classical hardware, they are inherently vulnerable to quantum attacks. Dilithium’s slightly larger key and signature sizes are a trade-off for quantum resistance \cite{ref0, ref4}.
    \item \textbf{Hardware Optimizations}: Recent work using vectorization (AVX-512) and FPGA designs has reduced the latency of Dilithium’s operations by significant margins (up to 40–45\%) \cite{ref9, ref6}. Such optimizations are critical for resource-constrained environments.
    \item \textbf{Side-Channel Considerations}: The simpler sampling and rejection procedures in Dilithium support constant-time implementations, reducing side-channel leakage risks.
    \item \textbf{Overall Trade-Offs}: In summary, while classical schemes may have smaller signatures (ECDSA) or similar signature lengths (RSA), their lack of quantum resistance makes them unsuitable for future secure communications. Dilithium provides a balanced solution with competitive performance in both software and hardware implementations.
\end{itemize}

\section{Conclusion}
This report demonstrates that although classical signature schemes such as RSA and ECDSA currently offer smaller key sizes and sometimes lower latencies under optimized conditions, their vulnerability to quantum attacks renders them unsuitable for long-term secure communications. CRYSTALS-Dilithium, by contrast, provides strong post-quantum security with acceptable space and time costs. Optimized hardware and vectorized software implementations show that its signing and verification speeds are competitive—and in some cases superior—when evaluated on overall system performance, energy, and area efficiency. Together with its design for constant-time operation, Dilithium is a compelling choice for next-generation secure communications in the post-quantum era.

\section*{References}
\begin{enumerate}
    \item CRYSTALS-Dilithium: Digital Signatures from Module Lattices. \cite{ref0}
    \item Performance and Applicability of Post-Quantum Digital Signature Algorithms in Resource-Constrained Environments. \cite{ref1}
    \item High-Performance Hardware Implementation of CRYSTALS-Dilithium. \cite{ref2}
    \item Comparative Evaluation of Post-Quantum Signature Schemes. \cite{ref3}
    \item CRYSTALS-Dilithium: Digital Signatures from Module Lattices (detailed paper). \cite{ref4}
    \item An Introduction to Post-Quantum Cryptography Algorithms. \cite{ref5}
    \item High-Performance Hardware Implementation of Lattice-Based Post-Quantum Digital Signature Schemes. \cite{ref6}
    \item Post-Quantum Authentication in TLS 1.3: A Performance Study. \cite{ref7}
    \item Crystals-Dilithium on ARMv8. \cite{ref8}
    \item Optimized Vectorization Implementation of CRYSTALS-Dilithium. \cite{ref9}
    \item NIST Post-Quantum Cryptography Standardization Process Report. \cite{ref10}
    \item Falcon: Fast Fourier Lattice-Based Signature Schemes. \cite{ref11}
    \item SPHINCS+ Signature Scheme Analysis and Benchmarking Studies. \cite{ref12}
    \item RSA Algorithm Performance Benchmarks (e.g., OpenSSL speed measurements). \cite{ref13}
    \item ECDSA Performance Benchmarks and Optimization Studies. \cite{ref14}
    \item BLISS: Lattice Signatures and Bimodal Gaussians. \cite{ref15}
    \item Lyubashevsky, ``Fiat--Shamir with Aborts: Applications to Lattice and Factoring-Based Signatures.'' \cite{ref16}
    \item An Introduction to Post-Quantum Cryptography (survey). \cite{ref17}
    \item FFT and NTT in Cryptography. \cite{ref18}
    \item Hardware Implementations of Post-Quantum Cryptography Algorithms (survey). \cite{ref19}
    \item Post-Quantum Digital Signatures: A Comparison (overview article). \cite{ref20}
\end{enumerate}

\end{document}